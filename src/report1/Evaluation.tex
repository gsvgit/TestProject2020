\section{Эксперментальное исследование}

Замеры производительности производились на ноутбуке под управлением операционной системы Ubuntu 18.04.5 LTS, с 16Гб оперативной памяти и процессором Intel Core i7-8550U, 1.80GHz.  

В экспериментах использовались случайные списки, сгенерированные с использованием стандартного генератора случайных чисел \verb|System.Random|.
В качестве значений использовались 32-битные целые числа.
Длины списков лежат в диапазоне от 10 тысяч до 2 миллионов элементов, шаг~---~30 тысяч элементов.

Эксперимент поставлен следующим образом. 
Для каждого значени длины 5 раз выполняются следующие шаги.
\begin{enumerate}
  \item Генерируется случайный список заданной длины.
  \item Сгенерированный список сортируется с использованием функции qSort~(листинг~\ref{lst:qSort}), при этом замеряется время, затраченное на сортировку. 
\end{enumerate}

Полученные результаты записываются в файл.
Затем то же самое повторяется для стандартной функции \verb|List.sort|.

Результаты замеров времени приведены на рисунке~\ref{fig:list_sort}.
Использован стандартный violin plot с указанием медианного значения времени для каждой длины.


\begin{figure}
	\centering
	\begin{subfigure}[b]{0.48\textwidth}
       \centering
	   \includegraphics[width=0.99\textwidth]{data/ListSort5ReleaseGC.pdf}
	   \caption{5 замеров}
	   \label{fig:list_sort_5}
	\end{subfigure}~
	\begin{subfigure}[b]{0.48\textwidth}
       \centering
	   \includegraphics[width=0.99\textwidth]{data/ListSort10ReleaseGC.pdf}
	   \caption{10 замеров}
	   \label{fig:list_sort_10}
	\end{subfigure}
	\caption{Сравнение произвоительности сортировок списка \texttt{List.sort} и \texttt{qSort} при разном количестве повторов одного замера}
	\label{fig:list_sort}	
\end{figure}


\begin{figure}
	\centering
    \includegraphics[width=0.99\textwidth]{data/ListSystemSort5ReleaseGCCharging2.pdf}
	\caption{Сравнение произвоительности сортировок списка \texttt{List.sort} при разных режимах сборки и запуска}
	\label{fig:list_sort_system}	
\end{figure}





\begin{figure}
	\centering
	\begin{subfigure}[b]{0.48\textwidth}
       \centering
	   \includegraphics[width=0.99\textwidth]{data/ListCustomSort5ReleaseGCCharging.pdf}
	   \caption{Активное использоване ноутбука во времмя замеров}
	   \label{fig:list_sort_5_trash}
	\end{subfigure}~
	\begin{subfigure}[b]{0.48\textwidth}
       \centering
	   \includegraphics[width=0.99\textwidth]{data/ListCustomSort5ReleaseGCCharging2.pdf}
	   \caption{Минимизация посторонней активности}
	   \label{fig:list_sort_5_clean}
	\end{subfigure}
	\caption{Зависимость качества графика от посторонней активности во время замеров}
	\label{fig:quality_of_measurements}	
\end{figure}
