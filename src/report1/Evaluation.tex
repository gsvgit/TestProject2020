\section{Эксперментальное исследование}

Замеры производительности производились на ноутбуке под управлением операционной системы Ubuntu 18.04.5 LTS, с 16Гб оперативной памяти и процессором Intel Core i7-8550U, 1.80GHz.  

В экспериментах использовались случайные списки, сгенерированные с использованием стандартного генератора случайных чисел \verb|System.Random|.
В качестве значений использовались 32-битные целые числа.
Длины списков лежат в диапазоне от 10 тысяч до 2 миллионов элементов, шаг~---~30 тысяч элементов.

Эксперимент поставлен следующим образом. 
Для каждого значени длины 5 раз выполняются следующие шаги.
\begin{enumerate}
  \item Генерируется случайный список заданной длины.
  \item Сгенерированный список сортируется с использованием функции qSort~(листинг~\ref{lst:qSort}), при этом замеряется время, затраченное на сортировку. 
\end{enumerate}

Полученные результаты записываются в файл.
Затем то же самое повторяется для стандартной функции \verb|List.sort|.

Результаты замеров времени приведены на рисунке~\ref{fig:list_sort}.
Использован стандартный violin plot с указанием медианного значения времени для каждой длины.

\begin{figure}
	\begin{center}
	   \includegraphics[width=0.95\textwidth]{data/ListSort.pdf}
	   \caption{Сравнение производительности системной сортировки и реализованной вручную сортировки Хоара для списка}
	   \label{fig:list_sort}
	\end{center}	
\end{figure}
