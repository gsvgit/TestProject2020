\section{Введение}

Сортировка коллекций --- одна из фундаментальных алгоритмимческих задач, изучению которой посвящено множество работ.
На текущий момент разработано огромное количество алгоритмов сортировки.
Часть из них являются классическими, такие как сортировка пузырьком или сортировка Хоара (или быстрая сортировка)~\cite{hoare1962quicksort}, другие жё применяются в достаточно специфических областях. Например, битоническая сортировка (bitonic sort)~\cite{nassimi1979bitonic} применяется, в основном, на массово-парарллельных устройствах, таких как GPGPU~\cite{10.5555/1882792.1882841}.
Стоит также отметить, что некоторые сортировки представляют, в основном, академический интерес.
Напрмиер, сортировка пузырьком, применяемая, как правило, для образовательных целей.

Для того, чтобы обоснованно выбирать алгоритмы сортировки для тех или иных ситуаций, необходимо понимать не только их теоретические свойства, такие как теоретическую оценку сложности, но и то, как те или иные алгоритмы и их реализации, которые могут отличаться для одного и того же алгоритма, ведут себя на практике.

В данной работе будет проведено экспериментальное сравнение двух сортировок для коллекции типа список (List) на платформе .NET.
А именно, мы сравним сортировку, предоставляемую стандартной библиотекой \verb|List.sort|, и реализованную вручную на языке программрования F\# сортировку Хоара. 